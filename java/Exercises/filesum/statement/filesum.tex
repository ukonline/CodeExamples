\documentclass[a4paper,11pt,final]{report}

\usepackage[latin1]{inputenc}
\usepackage[T1]{fontenc}
\usepackage[french]{babel}
\usepackage{lmodern}

\usepackage{graphicx}
\usepackage{tikz,pgf}
	\usetikzlibrary{arrows,calc}
\usepackage{color}
\usepackage{colortbl}
\usepackage{fancyhdr}
\usepackage{array}
\usepackage{diagbox}
\usepackage{listings}
\usepackage{url}
\usepackage{eurosym}
\usepackage{hyperref}

% Longueurs
\setlength\textheight{25cm}
\setlength\textwidth{17cm}
\setlength\oddsidemargin{0cm}
\setlength\evensidemargin{-1.5cm}
\setlength\parskip{0.2cm}
\setlength\topmargin{-15mm}
\setlength\parindent{0.0cm}

% Couleurs
\definecolor{lightgray}{gray}{0.95}
\definecolor{lightlightgray}{gray}{0.9}

% Param�trage des listings
\lstdefinelanguage{java}
{morekeywords=[1]{public,private,protected,class,interface,implements,extends,this,null,super,return,static,new,if,else,switch,case,while,for,do,break,continue,final,synchronized,try,catch,true,false,enum,assert,throw,throws,abstract,import},
morekeywords=[2]{double,float,long,int,short,byte,char,boolean,void},
morekeywords=[3]{String,System,BufferedReader,FileReader,IOException,BufferedWriter,FileWriter,ArithmeticException,Exception,List,ArrayList,Collections,Integer,Hashtable,Map,Runtime,StringBuilder},
morekeywords=[4]
{@Override},
morestring=[b]{"},
morecomment=[l]{//},
morecomment=[s]{/*}{*/},}

\lstset{basicstyle=\scriptsize\tt}
\lstset{keywordstyle=[1]\color[rgb]{0,0,1}\bfseries}
\lstset{keywordstyle=[2]\color[rgb]{0.6,0,0}\bfseries}
\lstset{keywordstyle=[3]\color[rgb]{0,0.6,0}\bfseries}
\lstset{identifierstyle=\color{black}}
\lstset{commentstyle=\color[rgb]{0,0.6,0.6}}
\lstset{stringstyle=\color[rgb]{0.44,0.47,1}}
\lstset{showstringspaces=false}
\lstset{showtabs=false}
\lstset{tabsize=3}
\lstset{extendedchars=true}
\lstset{breaklines=true}
\lstset{postbreak={}}
\lstset{breakautoindent=true}
\lstset{breakindent=0pt}
\lstset{xleftmargin=0.1cm}
\lstset{xrightmargin=1mm}
\lstset{frame=tblr,rulecolor=\color[gray]{.4}}
\lstset{captionpos=b}
\lstset{aboveskip=0.5cm,belowskip=0cm}
\lstset{numbers=left}
\lstset{numberstyle=\tiny}
\lstset{columns=fixed}
\lstset{escapechar=�}
\lstset{language=python}

\lstnewenvironment{stdout}{\lstset{numbers=none,language={},frame=single,backgroundcolor=\color{lightlightgray},framerule=2mm,rulecolor=\color{lightlightgray},xleftmargin=3mm,xrightmargin=3mm,belowskip=2mm}}{}

\lstnewenvironment{javaoutline}{\lstset{numbers=none,language=python,frame=none,xleftmargin=6mm,xrightmargin=1mm,belowskip=2mm}}{}

% Informations sur le document
\def\docsection{Exercices Java}
\def\doctitle{Sommes d'entiers d'un fichier}
\def\docauthor{S�bastien Comb�fis}
\def\docurl{http://www.ukonline.be/exercices/java/}

% Style du document
\urlstyle{sf}

% Ent�te et pied de page
\pagestyle{fancy}
\fancyhead{}
\fancyfoot{}
\fancyhead[RO,LE]{\textsc{\bfseries \doctitle}}
\fancyhead[LO,RE]{\textsc{\bfseries \docsection}}
\fancyfoot[RO,LE]{\bfseries \thepage~}
\fancyfoot[LO,RE]{\small \url{\docurl}}
\renewcommand{\headrulewidth}{1.5pt} 
\renewcommand{\footrulewidth}{0.5pt}

% Nouvelles commandes
\newcommand{\code}[1]{{\small\texttt{#1}}}

% = = = = = = = = = = = = = = = = = = = = = = = = = = = = = = = = = = = = = = = = = = = = = = = = = = = = = = = = =

% D�but du document
\begin{document}
\thispagestyle{plain} \noindent \rule{\textwidth}{1mm}
\begin{tabular}{m{3cm}p{13.15cm}}
\cellcolor{lightgray} \medskip \includegraphics[width=2.5cm]{images/uko-logo.png} \vspace{1.5cm} &
\cellcolor{lightgray} \begin{minipage}[t]{13cm}
\begin{flushright}
\vspace{-12mm} \hfill {\huge\textsc{\bfseries \docsection}} \\\vspace{0.6cm}
\hfill {\LARGE\sl \doctitle} \\\vspace{0.2cm}
\hfill {\sl par \docauthor} \\
\end{flushright}
\end{minipage}
\end{tabular} \\
\rule{\textwidth}{1mm}

%% - - - - - - - - - - - - - - - - - - - - - - - - - - - - - - - - - - - - - - - - - - - - - - - - - - - - - - - - - - - -

\section*{Contexte}

Cet exercice concerne la manipulation de fichiers textes. Soit un fichier texte qui contient, sur chacune de ses lignes, une s�quences de nombres entiers, s�par�s par exactement un espace. Le but de l'exercice consiste � �crire un programme qui va lire ce fichier et calculer, pour chaque ligne, la somme des nombres de la ligne. Toutes ces sommes doivent �tre �crites dans un autre fichier texte. Si une ligne du fichier d'origine n'est pas valide, le programme devra �crire \code{ERREUR} comme r�sultat.

Voici un exemple concret d'un fichier \code{input.txt} et du r�sultat \code{output.txt} que doit cr�er votre programme :

\begin{center}
	\scalebox{0.8}{\begin{tabular}{|>{\tt}p{3cm}|}
		\hline
		12 7 9 -8 \\
		6 0 0 2 \\
		-8 -2 12 SEB \\
		\\
		1 \\
		1 2 3 quatre \\
		89 98 \\
		\hline
		input.txt \\
		\hline
	\end{tabular}}
	\hspace{2cm}
	\scalebox{0.8}{\begin{tabular}{|>{\tt}p{3cm}|}
		\hline
		20 \\
		8 \\
		ERREUR \\
		0 \\
		1 \\
		ERREUR \\
		187 \\
		\hline
		output.txt \\
		\hline
	\end{tabular}}
\end{center}

%% - - - - - - - - - - - - - - - - - - - - - - - - - - - - - - - - - - - - - - - - - - - - - - - - - - - - - - - - - - - -

\section*{Impl�mentation}

D�finir une m�thode \code{computeSum} qui prend en param�tres deux chaines de caract�res repr�sentant des chemins vers des fichiers texte. Le premi�re param�tre est le nom du fichier source � lire et le second param�tre est le fichier destination � cr�er et qui va contenir les sommes calcul�es par le programme.

En cas d'erreurs de lecture ou d'�criture de fichier (\code{IOException}), la m�thode \code{computeSum} quitte imm�diatement le programme en affichant un message d'erreur.

\begin{lstlisting}[language=java]
public class FileSum
{
	private static void computeSum (String inFile, String outFile)
	{
		// � compl�ter
	}
	
	public static void main (String[] args)
	{
		computeSum ("input.txt", "output.txt");
	}
}
\end{lstlisting}

\end{document}